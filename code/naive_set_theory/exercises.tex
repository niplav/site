\beginsection Section 3, Exercise 1

It seems like there is no way one could use either insetting (putting
a given set into another set) and pairing or pairing on two different
inputs to obtain the same set. However, if one sees pairing the same set,
then pairing $\emptyset$ with $\emptyset$ would result in $\{\emptyset\}$,
which is also the result of insetting $\emptyset$.

\vskip 5mm

Proof:

\vskip 5mm

Insetting and pairing must have different results because insetting
will always result in a set with 1 element, and pairing will always
result in a set with 2 elements. Therefore, they can't be the same set.

Pairing the sets a, b and c, d can't result in the same set unless
$a=c$ and $b=d$ or $a=d$ and $b=c$. Otherwise, $\{a,b\}$ would contain
at least one element not in $\{c,d\}$.

\vskip 2mm
\settabs \+ \hskip 3mm& \cr
\vbox{
	\hrule width 3mm
	\+ \vrule height 3mm
	& \vrule height 3mm \cr
	\hrule width 3mm
}

\beginsection Section 4, Exercise 1

I am not exactly sure what I'm supposed to do here. I guess
``observe'' means ``prove'' here, so ``prove that the condition
has nothing to do with the set B''.

\vskip 5mm

Proof:

\vskip 5mm

$$ (A \cap B) \cup C = A \cap (B \cup C) \Leftrightarrow C \subset A $$

$$ (A \cup C) \cap (B \cup C) = A \cap (B \cup C) \Leftrightarrow C \subset A $$

$$ A \cup C = A \Leftrightarrow C \subset A $$

This is trivially true.

\vskip 2mm
\settabs \+ \hskip 3mm& \cr
\vbox{
	\hrule width 3mm
	\+ \vrule height 3mm
	& \vrule height 3mm \cr
	\hrule width 3mm
}

\beginsection Section 5, Some easy exercises

$$A-B=A \cap B'$$

Proof:

$$A-B=\{a | a \in A \land a \not \in B\}=\{a | a \in A \land a \in B'\}=A \cap B'$$


\vskip 2mm
\settabs \+ \hskip 3mm& \cr
\vbox{
	\hrule width 3mm
	\+ \vrule height 3mm
	& \vrule height 3mm \cr
	\hrule width 3mm
}

$$A \subset B \hbox{ if and only if } A-B=\emptyset$$

Proof:

$$ A \subset B \Leftrightarrow \forall a \in A: a \in B \Leftrightarrow \exists C: B=A \cup C \Leftrightarrow A-(A \cup C)=\emptyset \Leftrightarrow A-B=\emptyset $$

\vskip 2mm
\settabs \+ \hskip 3mm& \cr
\vbox{
	\hrule width 3mm
	\+ \vrule height 3mm
	& \vrule height 3mm \cr
	\hrule width 3mm
}

$$ A-(A-B)=A \cap B$$

Proof:

$$ A-(A-B)=A-(A \cap B')=A \cap (A \cap B')'=A \cap (A' \cup B)=A \cap A' \cup A \cap B=\emptyset \cup A \cap B=A\cap B$$

\vskip 2mm
\settabs \+ \hskip 3mm& \cr
\vbox{
	\hrule width 3mm
	\+ \vrule height 3mm
	& \vrule height 3mm \cr
	\hrule width 3mm
}

$$ A \cap (B-C)=(A \cap B)-(A \cap C) $$

Proof:

$$ (A \cap B)-(A \cap C)=(A \cap B) \cap (A \cap C)'=(A \cap B) \cap (A' \cup C')=(A \cap B \cap A') \cup (A \cap B \cap C')=A \cap B \cap C'=A \cap (B-C)$$

\vskip 2mm
\settabs \+ \hskip 3mm& \cr
\vbox{
	\hrule width 3mm
	\+ \vrule height 3mm
	& \vrule height 3mm \cr
	\hrule width 3mm
}

$$ A \cap B \subset (A \cap C) \cup (B \cap C') $$

Proof:

$$ A \cap B \subset (A \cap C) \cup (B \cap C') $$
$$ =((A \cap C) \cup B) \cap ((A \cap C) \cup C') $$
$$ =((A \cap C) \cup B) \cap ((A \cup C') \cap (C \cup C')) $$
$$ =((A \cap C) \cup B) \cap (A \cup C') $$
$$ =(A \cup B) \cap (C \cup B) \cap (A \cup C') $$

$A \cap B \subset (A \cup B) \cap (C \cup B) \cap (A \cup C')$ is true
because $A \subset (A \cup B)$ and $B \subset (C \cup B)$ and
$A \subset (A \cup C')$.

\vskip 2mm
\settabs \+ \hskip 3mm& \cr
\vbox{
	\hrule width 3mm
	\+ \vrule height 3mm
	& \vrule height 3mm \cr
	\hrule width 3mm
}

$$ (A \cup C) \cap (B \cup C') \subset A \cup B $$

Proof:

$$ (A \cup C) \cap (B \cup C') $$
$$ = ((A \cup C) \cap B) \cup ((A \cup C) \cap C') $$
$$ = ((A \cup C) \cap B) \cup A $$
$$ = (A \cap B) \cup (C \cap B) \cup A \subset A \cup B $$

This is the case because $(A \cap B) \cup (C \cap B) \subset B$ (since
intersections with $B$ are subsets of $B$), and the union with $A$
doesn't change the equation.

\vskip 2mm
\settabs \+ \hskip 3mm& \cr
\vbox{
	\hrule width 3mm
	\+ \vrule height 3mm
	& \vrule height 3mm \cr
	\hrule width 3mm
}

\beginsection Section 5, Exercise 1

To be shown: The power set of a set with n elements has $2^n$ elements.
Proof by induction.

\vskip 5mm

Proof:

\vskip 5mm

Induction base: The power set of the empty set contains 1 element:

$$|P(\emptyset)|=|{\emptyset}|=1=2^0=2^{|\emptyset|}$$

\vskip 5mm

Induction assumption:

$$|P(A)|=2^{|A|}$$

\vskip 5mm

Induction step:

\vskip 5mm

To be shown: $|P(A \cup \{a\}|=2*2^{|A|}=2^{|A|+1}$.

\vskip 2mm

$P(A \cup \{a\})$ contains two disjunct subsets: $P(A)$ and $N=\{\{a\}
\cup S | S \in P(A)\}$. Those are disjunct because every element in $N$
contains $a$ ($\forall n \in N: a \in n$), but there is no element of
$P(A)$ that contains $a$. Also, it holds that $P(A) \cup N=P(A \cup
\{a\})$, because elements in the power set can either contain $a$
or not, there is no middle ground.  It is clear that $|N|=|P(A)|$,
therefore $|P(A \cup \{a\})|=|P(A)|+|N|=2*|P(A)|=2*2^{|A|}=2^{|A|+1}$.

\vskip 2mm
\settabs \+ \hskip 3mm& \cr
\vbox{
	\hrule width 3mm
	\+ \vrule height 3mm
	& \vrule height 3mm \cr
	\hrule width 3mm
}

\beginsection Section 5, Exercise 2

To be shown:

$${\cal{P}}(E) \cap {\cal{P}}(F)={\cal{P}}(E \cap F)$$

Proof:

\vskip 5mm

If $S \in {\cal{P}}(E \cap F)$, then $\forall s \in S: s \in E \cap F$. Therefore,
$S \subset E$ and $S \subset F$ and thereby $S \in {\cal{P}}(E)$ and $S \in {\cal{P}}(F)$.
This means that $S \in {\cal{P}}(E) \cap {\cal{P}}(F)$.

If $S \in {\cal{P}}(E) \cap {\cal{P}}(F)$, then a very similar proof
can be written: $S \subset E$ and $S \subset F$, so $\forall s \in S:
s \in E$ and $\forall s \in S: s \in F$. Then $S \subset E \cap F$
and therefore $S \in {\cal{P}}(E \cap F)$.

\vskip 2mm
\settabs \+ \hskip 3mm& \cr
\vbox{
	\hrule width 3mm
	\+ \vrule height 3mm
	& \vrule height 3mm \cr
	\hrule width 3mm
}

\vskip 5mm

To be shown:

$${\cal{P}}(E) \cup {\cal{P}}(F)\subset{\cal{P}}(E \cup F)$$

Proof:

\vskip 5mm

If $S \in {\cal{P}}(E) \cup {\cal{P}}(F)$, then $S \in {\cal{P}}(E)
\Leftrightarrow S \subset E$ or $S \in {\cal{P}}(F) \Leftrightarrow S
\subset F$. Since it is true for any set $X$ that $S \subset E \Rightarrow
S \in {\cal{P}}(E \cup X)$, it is true that $S \in {\cal{P}}(E \cup F)$
(similar argumentation if $S \subset F$).

\vskip 2mm
\settabs \+ \hskip 3mm& \cr
\vbox{
	\hrule width 3mm
	\+ \vrule height 3mm
	& \vrule height 3mm \cr
	\hrule width 3mm
}

\vskip 5mm

A reasonable interpretation for the introduced notation:
If ${\cal{C}}={X_1, X_2, \dots, X_n}$, then

$$\bigcap_{X \in \cal{C}} X=X_1 \cap X_2 \cap \dots X_n$$

Similarly, if ${\cal{C}}={X_1, X_2, \dots, X_n}$, then

$$\bigcup_{X \in \cal{C}} X=X_1 \cup X_2 \cup \dots X_n$$

The symbol ${\cal{P}}$ still stands for the power set.

\vskip 5mm

To be shown:

$$\bigcap_{X \in \cal{C}} {\cal{P}}(X)={\cal{P}}(\bigcap_{X \in \cal{C}} X)$$

Proof by induction.

\vskip 5mm

Induction base:

$${\cal{P}}(E) \cap {\cal{P}}(F)={\cal{P}}(E \cap F)$$

Induction assumption:

$$\bigcap_{X \in \cal{C}} {\cal{P}}(X)={\cal{P}}(\bigcap_{X \in \cal{C}} X)$$

Induction step:

$${\cal{P}}(Y) \cap \bigcap_{X \in \cal{C}} {\cal{P}}(X)$$
$$={\cal{P}}(Y) \cap {\cal{P}}(\bigcap_{X \in \cal{C}} X)$$
$$={\cal{P}}(Y \cap \bigcap_{X \in \cal{C}} X)$$

The last step uses ${\cal{P}}(E) \cap {\cal{P}}(F)={\cal{P}}(E \cap F)$,
since $\bigcap_{X \in \cal{C}} X$ is also just a set.

\vskip 2mm
\settabs \+ \hskip 3mm& \cr
\vbox{
	\hrule width 3mm
	\+ \vrule height 3mm
	& \vrule height 3mm \cr
	\hrule width 3mm
}

\vskip 5mm

To be shown:

$$\bigcup_{X \in \cal{C}} {\cal{P}}(X) \subset{\cal{P}}(\bigcup_{X \in \cal{C}} X)$$

Proof by induction.

\vskip 5mm

Induction base:

$${\cal{P}}(E) \cup {\cal{P}}(F) \subset {\cal{P}}(E \cup F)$$

Induction assumption:

$$\bigcup_{X \in \cal{C}} {\cal{P}}(X) \subset {\cal{P}}(\bigcup_{X \in \cal{C}} X)$$

Induction step:

$${\cal{P}}(Y) \cup \bigcup_{X \in \cal{C}} {\cal{P}}(X)$$
$$\subset {\cal{P}}(Y) \cup {\cal{P}}(\bigcup_{X \in \cal{C}} X)$$
$$\subset {\cal{P}}(Y \cup \bigcup_{X \in \cal{C}} X)$$

The last step uses ${\cal{P}}(E) \cup {\cal{P}}(F) \subset {\cal{P}}(E
\cup F)$, since $\bigcup_{X \in \cal{C}} X$ is also just a set.

\vskip 2mm
\settabs \+ \hskip 3mm& \cr
\vbox{
	\hrule width 3mm
	\+ \vrule height 3mm
	& \vrule height 3mm \cr
	\hrule width 3mm
}

\vskip 5mm

To be shown:

$$\bigcup {\cal{P}}(E)=E$$

\vskip 5mm

Proof:

\vskip 5mm

$\forall X \in {\cal{P}}(E): X \subset E$. Furthermore, $E \in
{\cal{P}}(E)$.  Since $A \subset E \Rightarrow A \cup E=E$, it holds
that $E=\bigcup_{X \in {\cal{P}}(E)}=\bigcup {\cal{P}}(E)$.

\vskip 2mm
\settabs \+ \hskip 3mm& \cr
\vbox{
	\hrule width 3mm
	\+ \vrule height 3mm
	& \vrule height 3mm \cr
	\hrule width 3mm
}

\vskip 5mm

And "E is always equal to $\bigcup_{X \in {\cal{P}}(E)}$ (that is
$\bigcup {\cal{P}}(E)=E$), but that the result of applying ${\cal{P}}$
and $\bigcup$ to $E$ in the other order is a set that includes E as a
subset, typically a proper subset" (p. 21).

I am not entirely sure what this is supposed to mean. If it means
that we treat ${\cal{E}}$ as a collection, then $\forall X \in
{\cal{E}}:\bigcup_{E \in {\cal{E}}} E \subset X$. But that doesn't
mean that ${\cal{E}} \subset {\cal{P}}(\bigcup_{E \in {\cal{E}}}
E)$: If ${\cal{E}}=\{\{a,b\},\{b,c\}\}$, then $\bigcup_{E \in
{\cal{E}}} E=\{b\}$, and ${\cal{E}}=\{\{a,b\},\{b,c\}\} \not\subset
{\cal{P}}(\{b\})=\{\{b\},\emptyset\}$.

If we treat $E$ simply as a set, then $\bigcup E=E$, and it is of course
clear that $E \subset {\cal{P}}(E)$, as for all other subsets of $E$.

\vskip 2mm
\settabs \+ \hskip 3mm& \cr
\vbox{
	\hrule width 3mm
	\+ \vrule height 3mm
	& \vrule height 3mm \cr
	\hrule width 3mm
}

\beginsection Section 6, A non-trivial exercise

"find an intrinsic characterization of those sets of subsets of A that
correspond to some order in A"

Let ${\cal{M}} \subset {\cal{P}}({\cal{P}}(A))$ be the set of all possible
orderings of $A$. In the case of $A=\{a, b\}$, $\cal{M}$ would be
$\{\{\{a\}, \{a, b\}\}, \{\{b\}, \{a, b\}\}\}$.

Some facts about every element $M \in \cal{M}$:

$\bigcap M=\{min\}$, where $\{min\}$ is the smallest element in the ordering
$M$, the element in $M$ for which there is no other element $a \in M$
so that $a \subset \{min\}$ (which means that $\emptyset \not \in M$).

$\bigcup M=A$. $A$ must therefore be in $M$ and be the biggest element
(no element $a \in M$ so that $a \supset A$).

For all elements $m \in M$ except $\{min\}$ there exists at least one
element $n \in M$ so that $n \subset m$.

Similarly, for all elements $m \in M$ except $A$ there exists at least
one element $n \in M$ so that $n \supset m$.

For every $m \in M$ except $A$ and $\{min\}$, there exist two unique
elements $x,y \in M$ so that $m$ is the only set in $M$ for which it
is true that $x \subset m \subset y$.

For every $a \in A$, there must exist two sets $m,n \in M$ so that
$n=m \cup \{a\}$ (except for $min$). This means that the $|A|=|M|$,
the size of $A$ is the size of $M$.

These conditions characterise $\cal{M}$ intrinsically and are the solution
to the question.

\beginsection Section 6, Exercise 1

(i) To be shown: $(A \cup B) \times X=(A \times X) \cup (B \times X)$

\vskip 2mm

Proof:

\vskip 2mm

$(A \cup B) \times X=\{(e, x): e \in A \lor e \in B, x \in X\}=\{(e, x): e \in A, x \in X\} \cup \{(e, x): e \in B, x \in X\}=(A \times X) \cup (B \times X)$

\vskip 2mm
\settabs \+ \hskip 3mm& \cr
\vbox{
	\hrule width 3mm
	\+ \vrule height 3mm
	& \vrule height 3mm \cr
	\hrule width 3mm
}

\vskip 5mm

(ii) To be shown: $(A \cap B) \times (X \cap Y)=(A \times X) \cap (B \times Y)$

\vskip 2mm

Proof:

\vskip 2mm

$$(A \times X) \cap (B \times Y)=$$
$$\{(x,y), x \in A, y \in X\} \cap \{(v,w), v \in B, w \in Y\}=$$
$$\{(x,y), x \in A \land x \in B, y \in X \land y \in Y\}=$$
$$\{(x,y), x \in A \cap B, y \in X \cap Y\}=$$
$$(A \cap B) \times (X \cap Y)$$

\vskip 2mm
\settabs \+ \hskip 3mm& \cr
\vbox{
	\hrule width 3mm
	\+ \vrule height 3mm
	& \vrule height 3mm \cr
	\hrule width 3mm
}

\vskip 5mm

(iii) To be shown: $(A-B) \times X = (A \times X)-(B \times X)$

\vskip 2mm

Two-sided proof by contradiction:

\vskip 2mm

1. $(A-B)\times X \subset (A \times X)-(B \times X)$

Let $(u,v) \in (A-B) \times X$. Then $u \in (A-B)$, and $v \in X$. Suppose
$(u, v) \not \in (A \times X)-(B\times X)$. Then $(u,v) \in (A \times X)
\cap (B \times X)$. Then $(u,v) \in (A \cap B) \times (X \cap X)$. Then
$u \in A \cap B$ and $v \in X$. But if $u \in A \cap B$, then $u \not
\in A-B$! Contradiction.

\vskip 2mm

2. $ (A \times X)-(B \times X) \subset (A-B)\times X$

Let $(u,v) \in (A \times X)-(B\times X)$. Then $(u,v)\in(A \times X)$
and $(u,v)\not \in (B \times X)$. Because $v$ must be in $X$, and there
is no flexibility there, $u \not \in B$. Suppose $(u,v)\not \in (A-B)
\times X$. Since necessariliy $v \in X$, $u \not \in A-B$. But if $u
\not \in A-B$, $u$ must be an element of $A\cap B$. Then $u \in B$,
and there is a contradiction.

\vskip 2mm
\settabs \+ \hskip 3mm& \cr
\vbox{
	\hrule width 3mm
	\+ \vrule height 3mm
	& \vrule height 3mm \cr
	\hrule width 3mm
}

\beginsection Section 7, Exercise 1

Reflexive, but neither symmetric nor transitive (symmetry
violation: $(b,a)\not\in$, transitivity violation: $(a,c)\not\in$):
$\{(a,a),(a,b),(b,b),(b,c),(c,c)\}$

\vskip 2mm

Symmetric, but neither reflexive nor transitive (reflexivity
violation: $(a,a)\not\in$, transitivity violation: $(a,c)\not\in$):
$\{(a,b),(b,a),(b,c),(c,b)\}$

\vskip 2mm

Transitive, but neither reflexive nor symmetric (reflexivity
violation: $(a,a)\not\in$, symmetry violation: $(b,a)\not\in$):
$\{(a,b),(b,c),(a,c)\}$

\bye
